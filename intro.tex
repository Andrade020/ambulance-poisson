

\begin{abstract}
We consider models for spatiotemporal Poisson processes with some missing location data.
We discuss four models that make provision for missing location data, and their estimation.
The corresponding code is available on GitHub as an extension of LASPATED at {\url{https://github.com/vguigues/LASPATED}} in the Missing\_Data subdirectory.
We tested our models using the process of emergency call arrivals to an emergency medical service where the emergency reports often omit the location of the emergency.
We show the difference made by using models that make provision for missing location data.
\end{abstract}

\textbf{Keywords:} Spatiotemporal data, missing data, regularized likelihood estimator, regression.

\section{Introduction}

The study of spatiotemporal data
and the development of statistical models
for such data is important for many real-life applications. One such application is the 
management of a fleet of ambulances
of an Emergency Medical Service (EMS)
which requires studying the process of
emergency calls. These calls
have temporal information: the instant
of the call, spatial information: the location
of the call, and additional features such
as the priority of the call. This application
has been the object of a huge research work,
see for instance
\citep{fitz:73,gree:04,chai:72,jagt:17a,lisay:16,mcla:13b,mcla:13a,band:14,band:12,schm:12,lees:12,lees:11,ande:07,mayo:13,swov:73b,swov:73a,tore:71,dask:83,dask:81,berl:74,chur:74,Talarico2015} and references therein.
For that application, we developed
in \citep{guiklevhn2022}, \citep{ourheuristics23a}
\citep{websiteambrouting24}, optimization,
management, and visualization tools for
the management of the fleet of ambulances
of an EMS. These optimization models
are based on a rolling horizon approach 
\citep{guiguessagastiz2012,guiguessagastiz2010}
where each time a decision is taken a two-stage
stochastic program is solved using
scenarios of future calls. For 
the generation of these scenarios, we developed
statistical models and a library called LASPATED (Library for the Analysis of SPAtio TEmporal
Discrete data), 
see \citep{laspatedpaper, laspatedmanual}
for details. These statistical models assume that
for every call, we both have a time stamp
and the location of the call. 
Our models were tested with data from Rio
de Janeiro EMS.
However, we became aware, collaborating
with this EMS, that many emergency calls came without
the information of the location of the call. In this context,
the objective of this paper is twofold: (i)
develop statistical models for spatiotemporal
data that can deal with missing locations
and test these models with our historical
data of emergency calls to Rio de Janeiro EMS
and (ii) extend LASPATED software incorporating
in LASPATED the
calibration and test functions for the statistical
models we developed to deal with missing
locations.
Though there is already rich literature
on statistical models to deal with missing data, see for instance \citep{bookmd1, bookmd2,bookmd3},
our main contribution is to develop and test
extensions
of the specific models we proposed recently in
LASPATED
\citep{laspatedpaper, laspatedmanual}
to the case where some of the locations are not
present in the data. 
We also make our code
publicly available on GitHub and integrate it
in LASPATED, as an extension of our first version
of LASPATED.
The outline of the paper is the following.
In LASPATED, we proposed two types of models
(one without covariates and one with covariates)
based on a time and space discretization of the
data. If $\mathcal{T}$ is a partition of time
and $\mathcal{I}$ a partition of the studied
area into zones and if $\mathcal{C}$
is the set of arrival types, we assume that
the process of arrivals for time interval
$t \in \mathcal{T}$, zone $i \in \mathcal{I}$,
and arrival type $c \in \mathcal{C}$ is Poisson
with some intensity $\lambda_{c,i,t}$.
In Section \ref{sec:model1}, we propose
a modification of these models
(both non-regularized and regularized) as a splitting of Poisson processes with probabilities of missing the location either fixed or depending on arrival type and time.
In Section \ref{sec:model1m2}, we propose
another extension of the models from LASPATED
where the probability that an observation with missing location comes from a zone is proportional to the population of this zone.
In Section \ref{sec:num}, we present numerical
experiments applied to the aforementioned
data of emergency calls to Rio de Janeiro EMS,
comparing the performance
of our two models with simple estimators that
discard calls without locations. 
\if{
Due to space limitations, figures
of our numerical experiments
are reported in the online companion
\cite{laspatedmissingarxiv} of this
paper and  figure
numbers correspond to the number of 
the figures in this online 
companion.
}\fi


\section{Poisson Model with Estimated Probabilities of Missing Location Data}
\label{sec:model1}

\subsection{Maximum Likelihood Estimators}
\label{sec:model1ml}

Let $\mathcal{I}$ denote the set of zones that form a partition of the region that contains the data, let $\mathcal{T}$ denote a partition of time into time intervals, and let $\mathcal{C}$ denote the set of arrival types.
Let $\mathcal{D}_{t}$ denote the duration of time interval $t \in \mathcal{T}$.
Consider a Poisson process model in which the number of arrivals $X_{c,i,t,n}$ for arrival type $c$, zone $i$, time interval $t$, and observation $n$ is Poisson distributed with mean 
$\lambda_{c,i,t} \mathcal{D}_{t}$.
For every arrival type $c$ and time interval $t$, let
\begin{equation}
\label{sct}
S_{c,t} \ \ := \ \ \sum_{i \in \mathcal{I}} \lambda_{c,i,t}
\end{equation}
denote the sum of Poisson intensities over all zones.
Assume that for each arrival, there is a probability~$p$ that the location of the arrival is not reported (and probability $1-p$ that the location is reported).
The Bernoulli random variables indicating whether the location is reported or not are independent.
In this model, the probability of reporting the location does not depend on the zone, the arrival type, or the time interval.

Let $\mathcal{A}$ denote the set of arrivals, which is partitioned into $2$~subsets: $\mathcal{A} = \mathcal{A}^{0} \cup \mathcal{A}^{1}$.
The arrivals in $\mathcal{A}^{0}$ do not contain data about the location of the arrival, and the arrivals in $\mathcal{A}^{1}$ contain data about the location of the arrival.

Observation refers to a count of the number of arrivals in $\mathcal{A}^{0}$ for an arrival type~$c$ during a time interval~$t$ or to a count of the number of arrivals in $\mathcal{A}^{1}$ for an arrival type~$c$ in a zone~$i$ during a time interval~$t$.
For each $c \in \mathcal{C}$ and $t \in \mathcal{T}$, let $N^{0}_{c,t}$ denote the number of observations in $\mathcal{A}^{0}$ for arrival type~$c$ and time interval~$t$, and let these observations be indexed $n = 1,\ldots,N^{0}_{c,t}$.
For each $c \in \mathcal{C}$, $i \in \mathcal{I}$, and $t \in \mathcal{T}$, let $N^{1}_{c,i,t}$ denote the number of observations for arrival type~$c$, zone~$i$, and time interval~$t$, and let these observations be indexed $n = 1,\ldots,N^{1}_{c,i,t}$.
Assume that for each $c \in \mathcal{C}$, $i \in \mathcal{I}$, and $t \in \mathcal{T}$, it holds that $N^{1}_{c,i,t} = N^{0}_{c,t}$, and that observation $n \in \{1,\ldots,N^{1}_{c,i,t}\}$ corresponds to observation $n \in \{1,\ldots,N^{0}_{c,t}\}$, for example, these observations were made during the same week.
In what follows, we use the simplified notation $N_{c,t}$ for the common value of $N_{c,t}^{0}$ and $N_{c,i,t}^{1}$.

The Poisson process of all arrivals is the aggregation of the following two independent Poisson processes: 
\begin{itemize}
\item
the Poisson process of arrivals for which the location is reported, with corresponding random number of arrivals $Y_{c,i,t,n}$ for arrival type~$c$, zone~$i$, time interval~$t$, and observation $n$, and
\item
the Poisson process of arrivals for which the location is not reported with corresponding random number of arrivals $Z_{c,t,n}$ for arrival type~$c$, time interval~$t$, and observation~$n$.
\end{itemize}
Thus $Y_{c,i,t,n}$ is Poisson with mean $(1-p) \lambda_{c,i,t} \mathcal{D}_{t}$, and $Z_{c,t,n}$ is Poisson with mean $p S_{c,t} \mathcal{D}_{t}$.
For the process of calls with reported locations, let $M_{c,i,t,n}^{1}$ denote the number of arrivals for arrival type~$c$, zone~$i$, time interval~$t$, and observation~$n$.
Also, let 
$$
M_{c,\bullet,t,n}^{1} \ \ := \ \ \sum_{i \in \mathcal{I}} M_{c,i,t,n}^{1}
$$
denote the total number of arrivals over all zones for which the zone is reported, for arrival type~$c$, time interval~$t$, and observation~$n$, let
$$
M_{c,i,t,\bullet}^{1} \ \ := \ \ \sum_{n=1}^{N_{c,t}}  M_{c,i,t,n}^{1}
$$
denote the total number of arrivals over all observations for which the zone is reported, for arrival type~$c$, zone~$i$, and time interval~$t$, let
$$
M_{c,\bullet,t,\bullet}^{1} \ \ := \ \ \sum_{i \in \mathcal{I}} \sum_{n=1}^{N_{c,t}} M_{c,i,t,n}^{1}
$$
denote the total number of arrivals over all zones and observations for which the zone is reported, for arrival type~$c$, time interval~$t$, and let
$$
M^{1} \ \ := \ \ \sum_{c \in \mathcal{C}} \sum_{i \in \mathcal{I}} \sum_{t \in \mathcal{T}} \sum_{n=1}^{N_{c,t}} M_{c,i,t,n}^{1}
$$
denote the total number of all arrivals for which the location is reported.

For the process of calls with locations not reported, let $M_{c,t,n}^{0}$ denote the number of arrivals for arrival type~$c$, time interval~$t$, and observation~$n$.
Also, let
$$
M_{c,t,\bullet}^{0} \ \ := \ \ \sum_{n=1}^{N_{c,t}} M_{c,t,n}^{0}
$$
denote the total number of arrivals over all observations for which the zone is not reported, for arrival type~$c$ and time interval~$t$, and let
$$
M^{0} \ \ := \ \ \sum_{c \in \mathcal{C}} \sum_{t \in \mathcal{T}} \sum_{n=1}^{N_{c,t}} M_{c,t,n}^{0}
$$
the total number of all arrivals for which the zone is not reported.

Next we consider estimation of parameters $\lambda := (\lambda_{c,i,t}, c \in \mathcal{C}, i \in \mathcal{I}, t \in \mathcal{T})$ and $p$.

The likelihood function of the observed data is given by
\begin{align*}
L_{1}(\lambda,p) \ \ & = \ \ \prod_{c \in \mathcal{C}} \prod_{t \in \mathcal{T}} \prod_{n=1}^{N_{c,t}} \mathbb{P}(Z_{c,t,n} = M_{c,t,n}^{0}) \prod_{i \in \mathcal{I}} \mathbb{P}(Y_{c,i,t,n} = M_{c,i,t,n}^{1}) \\
& = \ \ \prod_{c \in \mathcal{C}} \prod_{t \in \mathcal{T}} \prod_{n=1}^{N_{c,t}} \exp(-p S_{c,t} \mathcal{D}_{t}) \frac{(p S_{c,t}\mathcal{D}_{t})^{M_{c,t,n}^{0}}}{M_{c,t,n}^{0}!} \prod_{i \in \mathcal{I}} \exp(-(1-p) \lambda_{c,i,t} \mathcal{D}_{t}) \frac{((1-p) \lambda_{c,i,t} \mathcal{D}_{t})^{M_{c,i,t,n}^{1}}}{M_{c,i,t,n}^{1}!}
\end{align*}
and therefore the log-likelihood function is given, up to a constant term that can be dropped for maximization, by
$$ 
\mathscr{L}_{1}(\lambda,p) \ \ = \ \ M^{1} \log(1-p) + M^{0} \log(p) + \sum_{c \in \mathcal{C}} \sum_{t \in \mathcal{T}} \left(-N_{c,t} S_{c,t} \mathcal{D}_{t} + M_{c,t,\bullet}^{0} \log(S_{c,t}) + \sum_{i \in \mathcal{I}} M_{c,i,t,\bullet}^{1} \log(\lambda_{c,i,t})\right).  
$$
The estimator $(\hat{\lambda},\hat{p})$ that maximizes $\mathscr{L}_{1}(\lambda,p)$ satisfies
$$
\frac{\partial \mathscr{L}_{1}(\hat{\lambda},\hat{p})}{\partial \lambda} \ \ = \ \ 0 \quad \mbox{and} \quad \frac{\partial \mathscr{L}_{1}(\hat{\lambda},\hat{p})}{\partial p} \ \ = \ \ 0
$$
which gives
\begin{equation}
\label{solvelp1}
\begin{array}{l}
\displaystyle \frac{M^{0}}{\hat{p}} - \frac{M^{1}}{1 - \hat{p}} \ \ = \ \ 0 \\
\displaystyle \frac{M_{c,t,\bullet}^{0}}{\hat{S}_{c,t}} - N_{c,t} \mathcal{D}_{t} + \frac{M_{c,i,t,\bullet}^{1}}{\hat{\lambda}_{c,i,t}} \ \ = \ \ 0, \quad \forall \ c \in \mathcal{C}, \ t \in \mathcal{T}, \ i \in \mathcal{I},
\end{array}
\end{equation}
where
\begin{equation}
\label{defshat}
\hat{S}_{c,t} \ \ = \ \ \sum_{i \in \mathcal{I}} \hat{\lambda}_{c,i,t}.
\end{equation}
It follows from~\eqref{solvelp1} that
\begin{equation}
\label{formp}
\hat{p} \ \ = \ \ \frac{M^{0}}{M^{1} + M^{0}}.
\end{equation}
Thus \eqref{formp} provides a very natural estimator of $p$ which is the empirical probability of not reporting the location, namely the total number $M^{0}$ of arrivals where the location is not reported divided by the total number $M^{1} + M^{0}$ of arrivals.
It also follows from~\eqref{solvelp1} that
\begin{equation}
\label{lambdaf1}
\hat{S}_{c,t} \ \ = \ \ \frac{M_{c,\bullet,t,\bullet}^{1} + M_{c,t,\bullet}^{0}}{N_{c,t} \mathcal{D}_{t}}.
\end{equation}
and
\begin{equation}
\label{lambdaf}
\hat{\lambda}_{c,i,t} \ \ = \ \ \frac{\hat{S}_{c,t} M_{c,i,t,\bullet}^{1}}{M_{c,\bullet,t,\bullet}^{1}}
\ \ = \ \ \underbrace{\left(\frac{M_{c,\bullet,t,\bullet}^{1} + M_{c,t,\bullet}^{0}}{M_{c,\bullet,t,\bullet}^{1}}\right)}_{1 / (1 - \hat{p}_{c,t})}
\underbrace{\left(\frac{M_{c,i,t,\bullet}^{1}}{N_{c,t} \mathcal{D}_{t}}\right)}_{\overline{\lambda}_{c,i,t} / \mathcal{D}_{t}}.
\end{equation}
Thus a very natural empirical estimator of intensities $\lambda_{c,i,t}$ is obtained.
Here,
\begin{equation}
\label{estimpct}
\hat{p}_{c,t} \ \ = \ \ \frac{M_{c,t,\bullet}^{0}}{M_{c,\bullet,t,\bullet}^{1} + M_{c,t,\bullet}^{0}}.
\end{equation}
denotes the empirical estimator of the probability~$p$ of not reporting for arrival type~$c$ and time interval~$t$, and
$$
\overline{\lambda}_{c,i,t} \ \ = \ \ \frac{M_{c,i,t,\bullet}^{1}}{N_{c,t}}
$$
denotes the empirical estimator of the mean number of arrivals $(1-p) \lambda_{c,i,t} \mathcal{D}_{t}$, for arrival type~$c$, zone~$i$, and time interval~$t$, for which the location is reported.
%The empirical estimation $\lambda^*_{c,i,t}$ of $\lambda_{c,i,t}$ therefore solves 
%$$
%(1 - \hat{p}_{c,t}) \lambda^{*}_{c,i,t} \mathcal{D}_{t} \ \ = \ \ \overline{\lambda}_{c,i,t} \ \ = \ \ \frac{M_{c,i,t,\bullet}^{1}}{N_{c,t}}
%$$
%and solving for $\lambda^{*}_{c,i,t}$, we get that this empirical estimation
%\begin{equation}
%\label{estlambda}
%\lambda^{*}_{c,i,t} \ \ = \ \ \left(\frac{1}{1-\hat{p}_{c,t}}\right) \left( \frac{M_{c,i,t,\bullet}^{1}}{N_{c,t} \mathcal{D}_{t}}\right)
%\ \ = \ \ \hat{\lambda}_{c,i,t}
%\end{equation}
%is the maximum likelihood estimate $\hat{\lambda}_{c,i,t}$ of $\lambda_{c,i,t}$ given by \eqref{lambdaf}.

Next consider a model in which the probabilities $p_{c,t}$ of not reporting the location depend on arrival type~$c$ and time period~$t$.
Let $p := (p_{c,t}, c \in \mathcal{C}, t \in \mathcal{T})$.
Now $Y_{c,i,t,n}$ is Poisson with mean $(1-p_{c,t}) \lambda_{c,i,t} \mathcal{D}_{t}$ and $Z_{c,t,n}$ is Poisson with mean $p_{c,t} S_{c,t} \mathcal{D}_{t}$.

The likelihood of the model is given by
\begin{align*}
& L_{2}(p,\lambda)    = \prod_{c \in \mathcal{C}} \prod_{t \in \mathcal{T}} \prod_{n=1}^{N_{c,t}} \mathbb{P}(Z_{c,t,n} = M_{c,t,n}^{0}) \prod_{i \in \mathcal{I}} \mathbb{P}(Y_{c,i,t,n} = M_{c,i,t,n}^{1}) \\
& =  \prod_{c \in \mathcal{C}} \prod_{t \in \mathcal{T}} \prod_{n=1}^{N_{c,t}} \exp(-p_{c,t} S_{c,t} \mathcal{D}_{t}) \frac{(p_{c,t} S_{c,t} \mathcal{D}_{t})^{M_{c,t,n}^{0}}}{M_{c,t,n}^{0}!} \prod_{i \in \mathcal{I}} \exp(-(1-p_{c,t}) \lambda_{c,i,t} \mathcal{D}_{t}) \frac{((1-p_{c,t}) \lambda_{c,i,t} \mathcal{D}_{t})^{M_{c,i,t,n}^{1}}}{M_{c,i,t,n}^{1}!}
\end{align*}
and therefore the log likelihood is given, up to a constant term that can be dropped for maximization, by
\begin{align*}
\mathscr{L}_{2}(\lambda,p) \ \ = \ \ & \sum_{c \in \mathcal{C}} \sum_{t \in \mathcal{T}} \left(-N_{c,t} S_{c,t} \mathcal{D}_{t} + M_{c,t,\bullet}^{0} \log(S_{c,t}) + \sum_{i \in \mathcal{I}} M_{c,i,t,\bullet}^{1} \log(\lambda_{c,i,t})\right) \\
& {} + \sum_{c \in \mathcal{C}} \sum_{t \in \mathcal{T}} \Big(M_{c,t,\bullet}^{0} \log(p_{c,t}) + M_{c,\bullet,t,\bullet}^{1} \log(1-p_{c,t})\Big).
\end{align*}
The pair $(\hat{\lambda},\hat{p})$ maximizing $\mathscr{L}_{2}(\lambda,p)$ solves
$$
\frac{\partial \mathscr{L}_{2}(\hat{\lambda},\hat{p})}{\partial \lambda} \ \ = \ \ 0 \quad \mbox{and} \quad \frac{\partial \mathscr{L}_{2}(\hat{\lambda},\hat{p})}{\partial p} \ \ = \ \ 0
$$
which gives
\begin{equation}
\label{solvelp}
\begin{array}{l}
\displaystyle \frac{M_{c,t,\bullet}^{0}}{\hat{p}_{c,t}} - \frac{M_{c,\bullet,t,\bullet}^{1}}{1-\hat{p}_{c,t}} \ \ = \ \ 0, \quad \forall \ c \in \mathcal{C}, \ t \in \mathcal{T} \\
\displaystyle \frac{M_{c,t,\bullet}^{0}}{\hat{S}_{c,t}} - N_{c,t} \mathcal{D}_{t} + \frac{M_{c,i,t,\bullet}^{1}}{\hat{\lambda}_{c,i,t}} \ \ = \ \ 0, \quad \forall \ c \in \mathcal{C}, \ i \in \mathcal{I}, \ t \in \mathcal{T},
\end{array}
\end{equation}
where $\hat{S}_{c,t}$ is given by~\eqref{defshat}.
Thus,
\begin{equation}
\label{estimpct2}
\hat{p}_{c,t} \ \ = \ \ \frac{M_{c,t,\bullet}^{0}}{M_{c,\bullet,t,\bullet}^{1} + M_{c,t,\bullet}^{0}}.
\end{equation}
and $\hat{\lambda}_{c,i,t}$ is still given by~\eqref{lambdaf}, as for the previous model.

Observe that when $\hat{p}_{c,t} = 0$ (that is, zones are reported for all arrivals), then $\hat{\lambda}_{c,i,t}$ equals the usual maximum likelihood estimator of a Poisson intensity, although $\mathscr{L}_{2}$ is not defined if $\hat{p}_{c,t} = 0$.
Also, $\hat{\lambda}_{c,i,t}$ cannot be estimated if $M_{c,\bullet,t,\bullet}^{1} = 0$, that is, if $\hat{p}_{c,t} = 1$.

To derive asymptotic confidence intervals for the estimators, we compute the Fisher information matrix of the models.
For the model with log-likelihood $\mathscr{L}_{2}$ and parameter $\theta := (\lambda,p)$, the Fisher information matrix has entries $-\mathbb{E}\left[\frac{\partial^2 \mathscr{L}_{2}(\theta)}{\partial \theta_{i} \partial \theta_{j}}\right]$.
Note that
\begin{align*}
-\frac{\partial \mathscr{L}_{2}(\lambda,p)}{\partial p_{c,t}} \ \ & = \ \ -\frac{M_{c,t,\bullet}^{0}}{p_{c,t}}+\frac{M_{c,\bullet,t,\bullet}^{1}}{1-p_{c,t}} \qquad \forall \ c \in \mathcal{C}, \ t \in \mathcal{T} \\
-\frac{\partial \mathscr{L}_{2}(\lambda,p)}{\partial \lambda_{c,i,t}} \ \ & = \ \ -\frac{M_{c,t,\bullet}^{0}}{S_{c,t}} + N_{c,t} \mathcal{D}_{t} - \frac{M_{c,i,t,\bullet}^{1}}{\lambda_{c,i,t}} \qquad \forall \ c \in \mathcal{C}, \ i \in \mathcal{I}, \ t \in \mathcal{T}.
\end{align*}
The second-order derivatives are therefore given by
\begin{align*}
-\frac{\partial^2 \mathscr{L}_{2}(\lambda,p)}{\partial p_{c,t}^2} \ \ & = \ \ \frac{M_{c,t,\bullet}^{0}}{p_{c,t}^2} + \frac{M_{c,\bullet,t,\bullet}^{1}}{(1-p_{c,t})^2} \qquad \forall \ c \in \mathcal{C}, \ t \in \mathcal{T} \\
-\frac{\partial^2 \mathscr{L}_{2}(\lambda,p)}{\partial \lambda_{c,i,t}^2} \ \ & = \ \ \frac{M_{c,t,\bullet}^{0}}{S_{c,t}^2} + \frac{M_{c,i,t,\bullet}^{1}}{\lambda_{c,i,t}^2} \qquad \forall \ c \in \mathcal{C}, \ i \in \mathcal{I}, \ t \in \mathcal{T} \\
-\frac{\partial^2 \mathscr{L}_{2}(\lambda,p)}{\partial \lambda_{c,i,t} \partial \lambda_{c,j,t}} \ \ & = \ \ \frac{M_{c,t,\bullet}^{0}}{S_{c,t}^2} \qquad \forall \ c \in \mathcal{C}, \ i,j \in \mathcal{I} \mbox{ with } i \neq j, \ t \in \mathcal{T} \\
-\frac{\partial^2 \mathscr{L}_{2}(\lambda,p)}{\partial \lambda_{c,i,t} \partial \lambda_{c',j,t'}} \ \ & = \ \ 0 \qquad \mbox{if } c \neq c' \mbox{ or } t \neq t' \\
-\frac{\partial^2 \mathscr{L}_{2}(\lambda,p)}{\partial p_{c',t'} \partial \lambda_{c,i,t}} \ \ & = \ \ 0 \qquad \forall \ c,c' \in \mathcal{C}, \ i \in \mathcal{I}, \ t,t' \in \mathcal{T} \\
-\frac{\partial^2 \mathscr{L}_{2}(\lambda,p)}{\partial p_{c,t} \partial p_{c',t'}} \ \ & = \ \ 0 \qquad \mbox{if } c \neq c' \mbox{ or } t \neq t'.
\end{align*}
Also,
\begin{align*}
\mathbb{E}[M_{c,t,\bullet}^{0}] \ \ & = \ \ p_{c,t}\mathcal{D}_{t} N_{c,t}S_{c,t} \\
\mathbb{E}[M_{c,\bullet,t,\bullet}^{1}] \ \ & = \ \ (1-p_{c,t})\mathcal{D}_{t} |\mathcal{I}|N_{c,t}\lambda_{c,i,t} \\
\mathbb{E}[M_{c,i,t,\bullet}^{1}] \ \ & = \ \ (1-p_{c,t})\mathcal{D}_{t} N_{c,t}\lambda_{c,i,t}.
\end{align*}
Thus the entries of the Fisher information matrix are given by
\begin{align*}
& \mathbb{E}\left[-\frac{\partial^2 \mathscr{L}_{2}(\lambda,p)}{\partial p_{c,t}^2}\right] \ \ = \ \ \frac{\mathcal{D}_{t} N_{c,t} S_{c,t}}{p_{c,t}} + \frac{\mathcal{D}_{t} |\mathcal{I}| N_{c,t} \lambda_{c,i,t}}{1-p_{c,t}} \qquad \forall \ c \in \mathcal{C}, \ t \in \mathcal{T} \\
& \mathbb{E}\left[-\frac{\partial^2 \mathscr{L}_{2}(\lambda,p)}{\partial \lambda_{c,i,t}^2}\right] \ \ = \ \ \frac{p_{c,t} \mathcal{D}_{t} N_{c,t}}{S_{c,t}} + \frac{(1-p_{c,t}) \mathcal{D}_{t} N_{c,t}}{ \lambda_{c,i,t}} \qquad \forall \ c \in \mathcal{C}, \ i \in \mathcal{I}, \ t \in \mathcal{T}\\
& \mathbb{E}\left[-\frac{\partial^2 \mathscr{L}_{2}(\lambda,p)}{\partial \lambda_{c,i,t} \partial \lambda_{c,j,t}}\right] \ \ = \ \ \frac{p_{c,t}\mathcal{D}_{t} N_{c,t}}{S_{c,t}} \qquad \forall \ c \in \mathcal{C}, \ i, j \in \mathcal{I} \mbox{ with } i \neq j, \ t \in \mathcal{T}.
\end{align*}
Variance estimates of the maximum likelihood estimators $\hat{\lambda}_{c,i,t}$ and $\hat{p}_{c,t}$ of $\lambda_{c,i,t}$ and $p_{c,t}$ are found on the diagonal of the inverse $I(\theta)^{-1}$ of the Fisher information matrix $I(\theta)$.
For example, with $\hat{V}^{\lambda}_{c,i,t}$ denoting the estimated variance of $\hat{\lambda}_{c,i,t}$, an asymptotic $1-\alpha$ confidence interval for $\lambda_{c,i,t}$ is given by
$$
\left[\hat{\lambda}_{c,i,t} - \Phi^{-1}(1-\alpha/2) \sqrt{\hat{V}^{\lambda}_{c,i,t}}, \ \hat{\lambda}_{c,i,t} + \Phi^{-1}(1-\alpha/2) \sqrt{\hat{V}^{\lambda}_{c,i,t}}\right]
$$
where $\Phi^{-1}(1-\alpha/2)$ denotes the $1-\alpha/2$ quantile of the standard normal distribution.
Similarly, a confidence interval for $p_{c,t}$ can be derived from $\hat{p}_{c,t}$ and $I(\theta)^{-1}$.
